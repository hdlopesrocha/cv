%!TEX TS-program = xelatex
\documentclass[]{friggeri-cv}
\addbibresource{bibliography.bib}
\usepackage{listings}
\usepackage{color}
\usepackage{graphicx}
\usepackage{mathtools}
\definecolor{pblue}{rgb}{0.13,0.13,1}
\definecolor{pgreen}{rgb}{0,0.5,0}
\definecolor{pred}{rgb}{0.9,0,0}
\definecolor{pgrey}{rgb}{0.46,0.45,0.48}

\lstset{language=Java,
  showspaces=false,
  showtabs=false,
  breaklines=true,
  showstringspaces=false,
  breakatwhitespace=true,
  commentstyle=\color{pgreen},
  keywordstyle=\color{pblue},
  stringstyle=\color{pred},
  basicstyle=\ttfamily
}

\begin{document}
\header{Henrique}{Rocha}
       {Telecommunications and Informatics}

% In the aside, each new line forces a line break
\begin{aside}
  \section{about}
    Henrique Duarte Lopes Rocha
    ~
    01210 Ferney-Voltaire
    France
    ~
    (+351) 938781922
    ~
    \href{mailto:hdlopesrocha@protonmail.com}{hdlopesrocha@protonmail.com}
    \href{https://github.com/hdlopesrocha}{GitHub}
    \href{http://www.youtube.com/user/hdlopesrocha}{YouTube}
  \section{languages}
    portuguese (native) \includegraphics[width=10pt]{pt}
    english (fluent) \includegraphics[width=10pt]{uk}
    french (A2) \includegraphics[width=10pt]{fr}
  \section{programming}
    {\color{red} $\varheartsuit$} Java
	(Android, PlayFramework, SpringBoot, LWJGL, MapReduce, GWT, AWS, JavaFX, Kurento Media Server, Jetty and Jersey), 
	{\color{green} $\varheartsuit$} Angular, {\color{green} $\varheartsuit$} Vue, React, C, C++ (OpenGL), C\# (XNA, Windows Forms, WPF, lpsolve), Grails, VB, Python, Scala, Scheme, 
	HTML, CSS (Bootstrap), JavaScript (jQuery, AJAX, WebGL, WebRTC), PHP (Wordpress), GLSL, HLSL.
\section{databases}
	DynamoDB, {\color{green} $\varheartsuit$} MongoDB, MySQL, Oracle, Neo4j, PostgreSQL, SQLite.
   \section{soft skills}
	Time Management
	Problem-Solving
	Team Player
	Self Confidence
	Adaptability
\end{aside}



\section{experience}
	\subsection{European Organization for Nuclear Research (CERN) \includegraphics[width=16pt]{ch} }  
		\begin{entrylist}

			\entry
			{01 2019\\(ongoing)}
			{Full-Stack development and Support}
			{FAP-BC-UI}
			{\emph{Maintenance, development and support of applications related to recruitment and document management. Experience with \underline{Vue}, \underline{React}, \underline{Elastic Search}, \underline{Prometheus}, \underline{OracleHR} and \underline{Activiti}} }


			\entry
			{09 2019\\(4 days)}
			{CERN Spring Campus}
			{Hamburg University of Technology , Germany}
			{\emph{Presentations about "Exploring Music Using the WebAudio API" and "Visualizing Music in 2D and 3D Using the Canvas API and WebGL"} (\href{https://hdlopesrocha.github.io/spring-campus-2019/dist/spring/}{GitHub})}

			\entry
			{06 2018\\(6 months)}
			{Full-Stack and Database development}
			{IMPACT}
			{\emph{Maintenance and optimization of an intervention management, planning and coordination tool. Experience with \underline{Grails} and \underline{HazelCast} (distributed cache)} }


			\entry
			{04 2018\\(4 days)}
			{CERN Spring Campus}
			{Riga Technical University, Latvia}
			{\emph{Presentations about "Real time communications between web-browsers using WebRTC" and "Collision detection for a massive amount of objects within a 3D environment"} }


			\entry
			{01 2018\\(6 months)}
			{Full-Stack and Mobile Development}
			{DigiWare}
			{\emph{Development of a work delivery tool for a warehouse at CERN. Experience with \underline{SpringBoot}, \underline{Ionic Framework}, \underline{Angular} and \underline{Hibernate} (with Oracle DB) } }

			\entry
			{10 2017\\(2 months)}
			{Front-end development and Business Intelligence}
			{APT}
			{\emph{Maintenance and front-end development of a cost analysis tool for CERN activities. Experience with \underline{Grails}, \underline{GWT} and \underline{Pentaho} } }

			\entry
			{12 2016\\(2 years)}
			{Applications Development}
			{PLAN}
			{\emph{Development and support for a planning tool for CERN activities. Experience with \underline{SpringBoot}, \underline{Handlebars}, \underline{Hibernate} (with Oracle DB). Used \underline{Event Sourcing (ES)} and \underline{Command Query Responsibility Segregation (CQRS)} design patterns. } }

			\entry
			{10 2016\\(2 months)}
			{Full-Stack Development}
			{PM-Support calendar}
			{\emph{Maintenance of a calendar application for support time management. Experience with \underline{SpringBoot} (integration with \underline{JIRA}), \underline{Thymeleaf} and \underline{Polymer}}}


		\end{entrylist}

	\subsection{Bullray-CIT \includegraphics[width=16pt]{pt}}  
		\begin{entrylist}
			\entry
			{01 2016\\(7 months)}
			{Applications Development}
			{}			
			{\emph{Development of a user interface for a bank's security device that is synchronized with servers. Micro-controller sensors information receival through \underline{USB} (\underline{org.usb4java}), user interface with \underline{JavaFX} and network camera discovery through nmap (using \underline{org.nmap4j}) } }

			\entry
			{11 2015\\(2 months)}
			{Android Development}
			{}
			{\emph{Optimization and bug resolution of an \underline{Android} application that uploads GPS, camera and microphone data in background.}}


			\entry
			{08 2015\\(2 months)}
			{Systems Development}
			{}
			{\emph{Development and configuration of a streaming server that supports receiving and recording video from authenticated users. I tried \underline{FFServer}, \underline{nginx with rtmp module} and \underline{Wowza Streaming Server}, the final decision was \underline{Kurento Media Server} with \underline{WebRTC}. }}

			\entry
			{02 2015\\(20 months)}
			{Full-Stack Web Development}
			{}
			{\emph{Development of a solution for a generic incident handling server, support for real time monitoring through \underline{WebSockets}. Experience with \underline{PlayFramework}, \underline{MongoDB} (with the official Java Driver), \underline{Maven}, \underline{jQuery} and \underline{Bootstrap}.}}

			\entry
			{11 2014\\(4 months)}
			{Full-Stack Web Development}
			{}
			{\emph{Development of an architecture for a chat application. Experience with \underline{Android}, \underline{PlayFramework}, \underline{MongoDB} (with Morphia), \underline{jQuery} and \underline{Bootstrap}. Message synchronization through HTTP long polling.}}
			
	
		\end{entrylist}

	\subsection{IST \includegraphics[width=16pt]{pt}}  
		\begin{entrylist}

			\entry
			{03 2015\\(1 year)}
			{Master Thesis \\\href{https://www.youtube.com/watch?v=TWfbcBKbseA}{watch it}}
			{IST}
			{\emph{Development of an interactive multi user video chat that supports recording, content overlay and a collaborative text editor. Used \underline{Kurento Media Server} for mixing streams into a single one and detecting QR codes. Experience with \underline{ot.js} for operation transformations (collaborative component). Other libraries used, \underline{jquery.qrcode}, \underline{vis.js}, \underline{bootstrap-typeahead}, \underline{codemirror.js} and \underline{adapter.js} }. Little experience with \underline{strophe.js} for communication between web browsers and \underline{XMPP} servers. }
			
			\entry
			{04 2014\\(6 months)}
			{Scientific Initiation Scholarship}
			{INESC}
			{\emph{Frontend Development for Provide Results's Bimk project. Experience with \underline{PlayFramework}, \underline{Revit API} and \underline{GIT}.}}
			
			\entry
			{09 2013\\(6 months)}
			{Scientific Initiation Scholarship}
			{CEG-IST}
			{\emph{Programming a Decision Analysis Software (MACBETH) with \underline{WPF}. Development of an algorithm to suggest judgements based on past decisions using \underline{lpsolve}.}}

		\end{entrylist}

\subsection{Other}  
	\begin{entrylist}
		\entry
		{08 2013\\(2 weeks)}
		{Summer Work}
		{LCG}
		{\emph{Frontend development for a Decision Analysis Software using \underline{ASP.NET}. }}
		\entry
		{11 2012}
		{Pizza Night Competition}
		{Microsoft Lisbon Experience}
		{\emph{2º place on Windows Phone Category} with \emph{Infinity Edge} game.}
		\entry
		{08 2006\\(1 month)}
		{Garbage Collector}
		{Jovens em Movimento, Oeiras}
		{Cleaning and maintenance of public spaces.}


		\end{entrylist}

\section{education}

\begin{entrylist}
   \entry
    {2013–2016}
    {Master of Science}
    {Instituto Superior Técnico - Taguspark}
    {Telecommunications and Informatics Engineering (Final Grade: 16 in 20)}
   \entry
    {2009–2014}
    {Bachelor of Science}
    {Instituto Superior Técnico - Taguspark}
    {Communication Networks Engineering (Final Grade: 13.2 in 20)}

\end{entrylist}



\section{applications}

\begin{entrylist}
  		\entry
		{2020}
		{VR music visualizer}
   		{\href{https://hdlopesrocha.github.io/vrMusic/}{Live demo}}
		{\emph{VR Music Visualizer using WebXR and WebGL} (\href{https://www.youtube.com/watch?v=DSdHLLsvcnM}{Video})}
  \entry
    {2015}
    {Bomb Raider}
    {\href{http://play.google.com/store/apps/details?id=pt.ist.bombraider}{Google Play}}
    {Bomb Raider is a simple singleplayer and multiplayer game where the goal is destroying all enemies in the map. I used \underline{OpenGLES} and \underline{WiFi-Direct} to develop this \underline{Android} game.}

  \entry
    {2013}
    {OpenGlobe}
    {\href{http://web.ist.utl.pt/ist168621/?page=10}{Instituto Superior Técnico}}
    {Modeling Earth surface with GPS devices and topographic collected data. It is also possible to predict satellites position through ephemeris data.}

  \entry
    {2012}
    {Infinity Edge (Windows Phone 7)}
    {\href{http://www.windowsphone.com/en-us/store/app/infinity-edge/05516463-3a89-4351-996e-62e5b4519aeb}{windows marketplace}}
    {First Person Shooter with spaceships made on my free time. I had experience with \underline{XNA} and \underline{HLSL}.}

  \entry
    {2011}
    {Era of Empires}
    {\href{http://web.ist.utl.pt/ist168621/?page=8}{Personal Project}}
    {3D environment generator and Real Time Strategy game engine using \underline{OpenGL}.}


\end{entrylist}

\section{interests}
    3D Studio Max, Photoshop, Game Development, Web Design, Network Security, Jogging, Kayaking, Karate, Yoga, Cycling, Wild camping, Survivalism and Crypto-Currencies.


%\newpage
%\section{grades}
%\subsection{Bachelor - Communication Networks Engineering}
%\input{grades_bachelor.tex}
%\subsection{Master - Telecommunications and Informatics Engineering}
%\input{grades_master.tex}


\begin{lstlisting}
return 0;
\end{lstlisting}
\end{document}
